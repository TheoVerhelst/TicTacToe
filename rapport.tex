\documentclass[a4paper]{article}

\usepackage[utf8]{inputenc}
\usepackage[T1]{fontenc}
\usepackage[francais]{babel}
\usepackage{times}
\usepackage[cm]{fullpage}
\usepackage{textcomp}
\author{Théo Verhelst}
\title{Rapport du projet 2 : \textit{Tic Tac Toe}\\Cours de Systèmes d'exploitation : INFO-F-201}

\begin{document}
\maketitle
\section{Introduction}
Ce document est le rapport relatif au projet du cours de Systèmes d'exploitation (\textit{INFO-F-201}) : \textit{Tic Tac Toe}.
Pour commencer, nous exposerons l'énoncé du projet ainsi que ses objectifs pédagogiques.
Ensuite seront discutés les choix effectués l'implémentation.
\subsection{Énoncé}
Le projet a pour but d'implémenter le jeu \textit{Tic Tac Toe}, également appelé \textit{Morpion}.
Le plateau de jeu consiste en une grille de $3\times3$ cases, et chacun des deux joueurs a pour objectif
de compléter une ligne, une colonne ou une diagonale avec le signe qui lui est assigné, l'ensemble des signes possibles
étant les lettres \textit{X} et \textit{O}.
Les joueurs jouent tour par tour, en placant un signe sur une case vide à chaque tour.\\
Le jeu doit être jouable en ligne de commande, et fonctionner par réseau, c'est à dire que
le joueur se confronte à un ordinateur distant unique appellé \textit{serveur}.
Par opposition, l'entité qui interagit à distance avec le serveur est appellée \textit{client}, ce terme
englobant l'utilisateur physique et son moyen de communication avec le serveur.\\
Le serveur doit pouvoir jouer avec un nombre non limité de clients.
\subsection{Objectifs}
L'intérêt du projet et de démontrer l'acquisition des connaisances vues en cours de Systèmes d'exploitation,
et plus précisément la communication réseau et les appels système, notamment le \textit{fork}.

\section{Implémentation}
\subsection{Modèle générale}
Le fonctionnement général de l'application consiste en deux programmes distincts, le \textit{client} et le \textit{serveur}.
Le serveur tourne en continu grâce à une boucle infinie, et accepte à chaque tour de boucle une nouvelle connexion entrante de la
part d'un client. La fonction gérant la connection avec le client et jouant le jeu est alors lancée dans un processus parallèle
au moyen de l'appel système \textit{fork}, tandis que le processus originel du serveur attend de nouveau une connexion entrante.

\subsection{Fonctionnement du client}
L'application client tente, à son lancement, de se connecter au serveur fourni en argument de ligne de commande.
Ensuite, une invite de début de partie est affichée, et si l'utilisateur confirme sa volonté de jouer, la partie commence.
À chaque tour de jeu, le client :
\begin{enumerate}
	\item reçoit de la part du serveur la grille de jeu;
	\item demande à l'utilisateur quelle case jouer;
	\item envoie la coordonée de la case jouée au serveur;
	\item reçoit la nouvelle situation de jeu suite à l'action du joueur;
	\item reçoit la nouvelle situation de jeu suite à l'action du serveur;
	\item recommence à la première étape;
\end{enumerate}
Chaque étape peut être interrompue par un problème de connexion, par la victoire d'un des joueurs, par un match null ou par la volonté du joueur
de quitter à la partie.


\subsection{Fonctionnement du serveur}
Le fonctionnement général du serveur déjà été décrit dans la section \textit{Modèle général}, cette section détaille plutôt
la procédure se chargeant de jouer une partie avec un client.
À chaque tour de jeu, le serveur:
\begin{enumerate}
	\item joue une case libre dans la grille;
	\item envoie la grille au client;
	\item reçoit de la part du client l'action de l'utilisateur;
	\item applique l'action de l'utilisateur sur la grille de jeu;
	\item envoie la grille au client;
	\item recommence à la première étape;
\end{enumerate}
Tout comme pour l'application cliente, chaque étape peut être interrompue par
un problème de connexion, par la victoire d'un des joueurs, par un match null ou par la volonté du joueur
de quitter à la partie.
\subsection{Conventions}
Les deux applications serveur et client utilisent une même procédure calculant l'état du jeu à partir de la grille courante,
qui permet de savoir s'il y a un match nul, un gagnant, ou si la partie n'est pas encore finie.
Cette fonction commune permet de tacitement fermer la connexion des deux côtés quand la partie est finie,
sans la nécessité pour le serveur d'envoyer au client un message de fin de partie.
\paragraph{}
Certaines des conventions de l'énoncé n'ont pas été appliqués et ont fait l'objet de définition statique
au moyen de la directive \textit{\#define}. Les simplifications en question sont principalement la taille
de la grille et le signe utilisé par le joueur ainsi que celui utilisé par le client. Cela permet une
plus grand généricité et une meilleure maintenabilité.
\subsection{Architecture du code}
Le code est divisé en 3 fichiers d'en-tête, ainsi que 3 fichiers d'implémentations associés.
Il y a bien évidemment le fichier concernant le serveur et le client, à savoir respectivement \textit{server.h} et \textit{client.h},
mais il y a aussi un fichier \textit{common.h} qui regroupe toutes les fonctions n'étant pas spécifiques au serveur ou au client.
Cela comprend notamment des méthodes de simplification de la communication réseau (se chargeant de la gestion des erreurs),
ainsi que les méthodes de manipulation de la grille de jeu qui interviennent dans les deux parties de l'architecture.
\end{document}
